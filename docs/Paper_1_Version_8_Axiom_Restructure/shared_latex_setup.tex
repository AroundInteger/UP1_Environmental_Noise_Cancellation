% SHARED LaTeX SETUP FOR UP2 & UP3 PARALLEL WRITING
% This file contains all shared packages, macros, and notation for consistency

% ===================================================================
% ESSENTIAL PACKAGES (Identical for both papers)
% ===================================================================
\usepackage[utf8]{inputenc}
\usepackage[T1]{fontenc}
\usepackage{amsmath,amsfonts,amssymb,amsthm}
\usepackage{mathtools}
\usepackage{geometry}
\usepackage{hyperref}
\usepackage{cleveref}
\usepackage{enumitem}
\usepackage{booktabs}
\usepackage{graphicx}
\usepackage{natbib}
\usepackage{array}
\usepackage{multicol}
\usepackage{algorithm}
\usepackage{algpseudocode}
\usepackage{xcolor}

% Page geometry (consistent across papers)
\geometry{margin=1in}

% ===================================================================
% SHARED THEOREM ENVIRONMENTS
% ===================================================================
\newtheorem{theorem}{Theorem}[section]
\newtheorem{principle}{Theorem}[section]
\newtheorem{lemma}[theorem]{Lemma}
\newtheorem{corollary}[theorem]{Corollary}
\newtheorem{proposition}[theorem]{Proposition}
\newtheorem{definition}[theorem]{Definition}
\newtheorem{remark}[theorem]{Remark}
\newtheorem{assumption}[theorem]{Assumption}
\newtheorem{example}[theorem]{Example}
\newtheorem{hypothesis}[theorem]{Hypothesis}

% ===================================================================
% UNIFIED MATHEMATICAL NOTATION (CRITICAL FOR CONSISTENCY)
% ===================================================================

% Core competitive measurement parameters
\newcommand{\deltaval}{\delta}           % Performance difference (μ_A - μ_B)/σ_A
\newcommand{\kappaval}{\kappa}           % Variance ratio σ_B/σ_A  
\newcommand{\etaval}{\eta}               % Environmental noise ratio σ_η/σ_A

% Mahalanobis distance variants
\newcommand{\DM}{D_{\text{M}}}           % Basic Mahalanobis distance
\newcommand{\DMenv}{D_{\text{M}}^{(\text{env})}} % Environmental Mahalanobis distance

% Quadrant notation (consistent Q1-Q4 across papers)
\newcommand{\QOne}{Q_1}                  % Optimal quadrant
\newcommand{\QTwo}{Q_2}                  % Transitional quadrant  
\newcommand{\QThree}{Q_3}                % Inverse quadrant
\newcommand{\QFour}{Q_4}                 % Crisis/Extinct quadrant

% Performance measures
\newcommand{\Sth}{S_{\text{th}}}         % Theoretical separability
\newcommand{\Semp}{S_{\text{emp}}}       % Empirical separability
\newcommand{\Sabs}{S_{\text{abs}}}       % Absolute measure separability
\newcommand{\Srel}{S_{\text{rel}}}       % Relative measure separability

% Information and effect size measures
\newcommand{\Icontent}{I}                % Information content
\newcommand{\effectsize}{d}              % Effect size (Cohen's d relationship)

% Evolutionary fitness function (UP3 specific but referenced in UP2)
\newcommand{\Fitness}{F(\deltaval, \kappaval)} % Evolutionary fitness function
\newcommand{\FitnessQ}[1]{F_{Q_{#1}}}    % Quadrant-specific fitness

% Environmental integration formula (UP2 specific but referenced in UP3)
\newcommand{\SNRimprovement}{\text{SNR}_{\text{improvement}}}
\newcommand{\SNRformula}{\frac{1}{\sqrt{1 + \frac{2\etaval^2}{1 + \kappaval^2}}}}

% Statistical operators
\DeclareMathOperator{\sign}{sign}
\DeclareMathOperator{\Var}{Var}
\DeclareMathOperator{\Cov}{Cov}
\DeclareMathOperator{\E}{\mathbb{E}}
\DeclareMathOperator{\Prob}{\mathbb{P}}
\DeclareMathOperator*{\argmax}{arg\,max}
\DeclareMathOperator*{\argmin}{arg\,min}

% ===================================================================
% SHARED COLOR SCHEME (For consistent figures)
% ===================================================================
\definecolor{Q1color}{RGB}{0, 128, 0}    % Green for Q1 (Optimal)
\definecolor{Q2color}{RGB}{255, 165, 0}  % Orange for Q2 (Transitional)
\definecolor{Q3color}{RGB}{255, 0, 0}    % Red for Q3 (Inverse)
\definecolor{Q4color}{RGB}{128, 128, 128} % Gray for Q4 (Extinct)
\definecolor{envcolor}{RGB}{0, 0, 255}   % Blue for environmental effects

% ===================================================================
% CROSS-PAPER REFERENCE COMMANDS
% ===================================================================
% These will be updated with actual paper references once submitted
\newcommand{\paperone}{UP1}          % Reference to empirical relative measures paper
\newcommand{\papertwo}{UP2}        % Self-reference for UP2
\newcommand{\paperthree}{UP3}      % Self-reference for UP3
\newcommand{\paperfour}{UP4}       % Forward reference to temporal dynamics

% Cross-paper equation references (to be updated)
\newcommand{\DMformula}{\DM = \frac{|\deltaval|}{\sqrt{1 + \kappaval^2}}}
\newcommand{\DMenvformula}{\DMenv = \frac{|\deltaval|}{\sqrt{1 + \kappaval^2 + 2\etaval^2}}}
\newcommand{\Fitnessformula}{\Fitness = \begin{cases} 
    \deltaval \times (1 - \kappaval) & \text{if } \kappaval < 1 \\
    \deltaval \times (\kappaval - 1) & \text{if } \kappaval > 1 
\end{cases}}

% ===================================================================
% SHARED MACROS FOR RESULTS PRESENTATION
% ===================================================================
\newcommand{\correlationUPTwo}{r = 0.973}        % UP2 SVM correlation
\newcommand{\correlationUPThree}{r = 0.912}      % UP3 theory-data correlation
\newcommand{\SNRresult}{53.2\%}                  % UP2 SNR improvement result
\newcommand{\QFourextinction}{0}                 % UP3 Q4 extinction result (0 datasets)
\newcommand{\parametercount}{644}                % Total parameter combinations tested
\newcommand{\domaincount}{5}                     % Cross-domain validation count

% ===================================================================
% FIGURE AND TABLE FORMATTING (Consistent style)
% ===================================================================
% Standard figure width for consistency
\newcommand{\figwidth}{0.8\textwidth}
\newcommand{\smallfigwidth}{0.6\textwidth}

% Table formatting for results
\newcommand{\resultscaption}[1]{\caption{#1}}
\newcommand{\validationtable}{\begin{tabular}{lccc} \toprule}
%\newcommand{\endvalidationtable}{\bottomrule \end{tabular}}

% ===================================================================
% JOURNAL-SPECIFIC FORMATTING FLAGS
% ===================================================================
% These can be toggled based on target journal requirements
\newif\ifNatureformat\Natureformatfalse       % Nature family journals
\newif\ifIEEEformat\IEEEformatfalse           % IEEE journals
\newif\ifJMLRformat\JMLRformatfalse           % JMLR format
\newif\ifPLOSformat\PLOSformatfalse           % PLOS family

% ===================================================================
% SHARED ABSTRACT COMPONENTS
% ===================================================================
% Keywords that should appear in both papers for consistency
\newcommand{\sharedkeywords}{competitive measurement, Mahalanobis distance, asymmetric analysis, quadrant classification}

% Common background statement for both papers
\newcommand{\competitivemeasurement}{Competitive measurement between entities A and B represents a fundamental challenge across domains from financial analysis to clinical research}

% Common framework reference
\newcommand{\frameworkfoundation}{Building on the empirical success of relative measures $R = X_A - X_B$ established in \paperone}

% ===================================================================
% VALIDATION RESULT FORMATTING
% ===================================================================
\newcommand{\validationresult}[3]{%
  \textbf{#1}: #2 (#3)%
}

\newcommand{\quadrantresult}[4]{%
  \textbf{#1} ($\deltaval #2 0, \kappaval #3 1$): #4%
}

% ===================================================================
% MANUSCRIPT STRUCTURE CONSISTENCY
% ===================================================================
% Ensure both papers follow the same high-level structure
\newcommand{\abstractlength}{150 words}
\newcommand{\introductionlength}{1.5 pages}
\newcommand{\theorylength}{3 pages}
\newcommand{\resultslength}{2.5 pages}
\newcommand{\discussionlength}{1 page}

% ===================================================================
% DEBUGGING AND CONSISTENCY CHECKS
% ===================================================================
% Commands to verify cross-paper consistency during writing
\newcommand{\consistencycheck}[1]{\textcolor{red}{CHECK: #1}}
\newcommand{\crossref}[1]{\textcolor{blue}{CROSSREF: #1}}
\newcommand{\needsupdate}[1]{\textcolor{orange}{UPDATE: #1}}

% Remove these in final version by redefining as empty
% \renewcommand{\consistencycheck}[1]{}
% \renewcommand{\crossref}[1]{}
% \renewcommand{\needsupdate}[1]{}

% ===================================================================
% AUTHOR INFORMATION TEMPLATE
% ===================================================================
\newcommand{\authorone}{M.R. Brown\thanks{Corresponding author. Email: m.r.brown@swansea.ac.uk}}
\newcommand{\authortwo}{Author 2}
\newcommand{\authorthree}{Author 3}

\newcommand{\affiliationone}{Department of Applied Mathematics, Swansea University}
\newcommand{\affiliationtwo}{Department of Statistics, University Name}  
\newcommand{\affiliationthree}{Department of Business Analytics, University Name}

% ===================================================================
% USAGE INSTRUCTIONS
% ===================================================================
% Include this file in both UP2 and UP3 manuscripts with:
% % ===================================================================
% SHARED LATEX SETUP FOR CORRELATION-BASED FRAMEWORK PAPER
% ===================================================================

% Packages
\usepackage[utf8]{inputenc}
\usepackage[T1]{fontenc}
\usepackage{amsmath,amsfonts,amssymb,amsthm}
\usepackage{graphicx}
\usepackage{booktabs}
\usepackage{array}
\usepackage{multirow}
\usepackage{float}
\usepackage{geometry}
\usepackage{setspace}
\usepackage{natbib}
\usepackage{hyperref}
\usepackage{xcolor}
\usepackage{tikz}
\usepackage{pgfplots}

% Page setup
\geometry{a4paper,margin=1in}
\onehalfspacing

% Author information
\newcommand{\authorone}{Rowan Brown}
\newcommand{\affiliationone}{University of [Institution]}

% Custom commands
\newcommand{\competitivemeasurement}{Competitive measurement}
\newcommand{\snr}{SNR}
\newcommand{\correlation}{\rho}
\newcommand{\variance}{\kappa}

% Theorem environments
\theoremstyle{definition}
\newtheorem{axiom}{Axiom}
\newtheorem{theorem}{Theorem}
\newtheorem{lemma}{Lemma}
\newtheorem{corollary}{Corollary}
\newtheorem{definition}{Definition}
\newtheorem{example}{Example}
\newtheorem{remark}{Remark}

% Custom environments
\newenvironment{proof}{\noindent\textbf{Proof.}}{\hfill$\square$}

% Figure and table captions
\captionsetup{font=small,labelfont=bf}

% Hyperref setup
\hypersetup{
    colorlinks=true,
    linkcolor=blue,
    urlcolor=blue,
    citecolor=red
}

% 
% This ensures perfect consistency in:
% - Mathematical notation (δ, κ, η, D_M, etc.)
% - Quadrant references (Q1, Q2, Q3, Q4)
% - Cross-paper citations
% - Figure and table formatting
% - Color schemes for visualizations
% - Statistical result presentation
%
% Update the cross-references and journal formatting flags as needed
% for specific submission targets.



% ===================================================================
% PAPER-SPECIFIC METADATA (Add to existing file)
% ===================================================================
\newcommand{\UPOneTitle}{Environmental Noise Cancellation in Competitive Measurement}
\newcommand{\UPTwoTitle}{Asymmetric Mahalanobis Framework for Competitive Measurement}
\newcommand{\UPThreeTitle}{Evolutionary Extinction Theory in Competitive Measurement}
\newcommand{\UPFourTitle}{Temporal Dynamics in Competitive Measurement}

% Abstract word counts for consistency checking
\newcommand{\abstractwordcount}{150}
\newcommand{\checkabstractlength}[1]{%
  \immediate\write16{Abstract length: #1 words (target: \abstractwordcount)}%
}

% ===================================================================
% CROSS-PAPER CITATION COMMANDS
% ===================================================================
\newcommand{\citeUPOne}{\cite{UP1}}      % Will be updated with actual citations
\newcommand{\citeUPTwo}{\cite{UP2}}
\newcommand{\citeUPThree}{\cite{UP3}}
\newcommand{\citeUPFour}{\cite{UP4}}

% Equation references across papers
\newcommand{\eqnSNRimprovement}{Equation (17) in \paperone}
\newcommand{\eqnMahalanobis}{Equation (12) in \papertwo}
\newcommand{\eqnFitness}{Equation (8) in \paperthree}

% ===================================================================
% RESULT FORMATTING CONSISTENCY
% ===================================================================
\newcommand{\resultformat}[2]{\textbf{#1:} #2}
\newcommand{\correlationformat}[1]{$r = #1$}
\newcommand{\percentformat}[1]{#1\%}
\newcommand{\significanceformat}[1]{$p < #1$}

% ===================================================================
% FIGURE CAPTION TEMPLATES
% ===================================================================
\newcommand{\quadrantfigcaption}{Four-quadrant classification of competitive scenarios showing the relationship between performance difference ($\deltaval$) and variance asymmetry ($\kappaval$)}

\newcommand{\validationfigcaption}{Theoretical vs. empirical validation showing strong correlation between framework predictions and observed results across multiple domains}