\appendix
\section{Mathematical Proofs}

This appendix provides rigorous mathematical proofs for the key theoretical results presented in the main paper. These proofs establish the mathematical foundation for the correlation-based signal enhancement framework.

\subsection{Proof of Axiom 4: Statistical Optimality}

\textbf{Theorem:} Under the correlation-based measurement model, the relative measure $R = X_A - X_B$ is the Minimum Variance Unbiased Estimator (MVUE) of the performance difference $\mu_A - \mu_B$.

\textbf{Proof:}

Consider the measurement model:
\begin{align}
X_A &= \mu_A + \varepsilon_A \\
X_B &= \mu_B + \varepsilon_B
\end{align}

where $\varepsilon_A \sim N(0, \sigma_A^2)$, $\varepsilon_B \sim N(0, \sigma_B^2)$, and $\text{Cov}(\varepsilon_A, \varepsilon_B) = \rho\sigma_A\sigma_B$.

\textbf{Step 1: Unbiasedness}
The relative measure $R = X_A - X_B$ is an unbiased estimator of $\mu_A - \mu_B$:
\begin{align}
E[R] &= E[X_A - X_B] \\
&= E[X_A] - E[X_B] \\
&= \mu_A - \mu_B
\end{align}

\textbf{Step 2: Variance Calculation}
The variance of $R$ is:
\begin{align}
\text{Var}(R) &= \text{Var}(X_A - X_B) \\
&= \text{Var}(X_A) + \text{Var}(X_B) - 2\text{Cov}(X_A, X_B) \\
&= \sigma_A^2 + \sigma_B^2 - 2\rho\sigma_A\sigma_B
\end{align}

\textbf{Step 3: Cramér-Rao Lower Bound}
For the parameter $\theta = \mu_A - \mu_B$, the Fisher Information is:
\begin{align}
I(\theta) &= \frac{1}{\text{Var}(R)} = \frac{1}{\sigma_A^2 + \sigma_B^2 - 2\rho\sigma_A\sigma_B}
\end{align}

The Cramér-Rao Lower Bound is:
\begin{align}
\text{CRLB} &= \frac{1}{I(\theta)} = \sigma_A^2 + \sigma_B^2 - 2\rho\sigma_A\sigma_B
\end{align}

\textbf{Step 4: Efficiency}
Since $\text{Var}(R) = \text{CRLB}$, the estimator $R$ achieves the Cramér-Rao Lower Bound and is therefore efficient.

\textbf{Step 5: Completeness and Sufficiency}
Under the normal distribution assumption, $R$ is a complete and sufficient statistic for $\mu_A - \mu_B$. By the Lehmann-Scheffé theorem, $R$ is the unique MVUE.

\textbf{Conclusion:} $R = X_A - X_B$ is the MVUE of $\mu_A - \mu_B$ under the correlation-based measurement model.

\subsection{Derivation of Signal Enhancement Factor (SEF)}

\textbf{Theorem:} The Signal Enhancement Factor for correlation-exploiting relative measurement compared to independent measurement is given by:
\begin{equation}
\text{SEF} = \frac{\text{SNR}_R}{\text{SNR}_{\text{independent}}} = \frac{1 + \kappa}{1 + \kappa - 2\sqrt{\kappa}\rho}
\end{equation}

where $\kappa = \sigma_B^2/\sigma_A^2$ and $\rho$ is the correlation coefficient.

\textbf{Proof:}

\textbf{Step 1: Independent Measurement SNR}
When using both measurements independently (correlation ignored):
\begin{align}
\text{SNR}_{\text{independent}} &= \frac{(\mu_A - \mu_B)^2}{\sigma_A^2 + \sigma_B^2} = \frac{\delta^2}{\sigma_A^2 + \sigma_B^2}
\end{align}

This represents the SNR when treating $X_A$ and $X_B$ as independent measurements, where $\text{Var}(X_A - X_B) = \sigma_A^2 + \sigma_B^2$.

\textbf{Step 2: Correlated Measurement SNR}
For relative measurement using $R = X_A - X_B$ with correlation exploited:
\begin{align}
\text{SNR}_R &= \frac{(\mu_A - \mu_B)^2}{\text{Var}(R)} = \frac{\delta^2}{\sigma_A^2 + \sigma_B^2 - 2\rho\sigma_A\sigma_B}
\end{align}

\textbf{Step 3: Signal Enhancement Factor}
\begin{align}
\text{SEF} = \frac{\text{SNR}_R}{\text{SNR}_{\text{independent}}} &= \frac{\delta^2/(\sigma_A^2 + \sigma_B^2 - 2\rho\sigma_A\sigma_B)}{\delta^2/(\sigma_A^2 + \sigma_B^2)} \\
&= \frac{\sigma_A^2 + \sigma_B^2}{\sigma_A^2 + \sigma_B^2 - 2\rho\sigma_A\sigma_B}
\end{align}

\textbf{Step 4: Substitution}
Substituting $\kappa = \sigma_B^2/\sigma_A^2$ and $\sigma_B = \sqrt{\kappa}\sigma_A$:
\begin{align}
\text{SEF} = \frac{\text{SNR}_R}{\text{SNR}_{\text{independent}}} &= \frac{\sigma_A^2 + \kappa\sigma_A^2}{\sigma_A^2 + \kappa\sigma_A^2 - 2\rho\sigma_A\sqrt{\kappa}\sigma_A} \\
&= \frac{\sigma_A^2(1 + \kappa)}{\sigma_A^2(1 + \kappa - 2\rho\sqrt{\kappa})} \\
&= \frac{1 + \kappa}{1 + \kappa - 2\rho\sqrt{\kappa}}
\end{align}

\textbf{Conclusion:} The Signal Enhancement Factor (SEF) is derived as stated. This formula quantifies the improvement achieved by exploiting correlation between competitors compared to treating their measurements as independent, following established enhancement factor conventions in signal processing literature.

\subsection{Proof of Scale Independence}

\textbf{Theorem:} The Signal Enhancement Factor (SEF) is independent of the absolute scale of the performance difference $\delta = |\mu_A - \mu_B|$.

\textbf{Proof:}

From the Signal Enhancement Factor formula:
\begin{align}
\text{SEF} &= \frac{1 + \kappa}{1 + \kappa - 2\sqrt{\kappa}\rho}
\end{align}

The $\delta^2$ terms cancel out in the ratio calculation, leaving only:
- $\kappa = \sigma_B^2/\sigma_A^2$ (variance ratio)
- $\rho$ (correlation coefficient)

\textbf{Implications:}
1. The SEF is independent of the absolute performance difference
2. Only the relative variance structure ($\kappa$) and correlation ($\rho$) matter
3. The framework applies universally across different measurement scales
4. Identical SEF values can be achieved regardless of domain-specific units

\subsection{Correlation-Based Variance Reduction Proof}

\textbf{Theorem:} When $\rho > 0$, the variance of the relative measure $R$ is reduced compared to the sum of individual variances.

\textbf{Proof:}

\textbf{Step 1: Variance of R}
\begin{align}
\text{Var}(R) &= \sigma_A^2 + \sigma_B^2 - 2\rho\sigma_A\sigma_B
\end{align}

\textbf{Step 2: Comparison with Sum of Variances}
\begin{align}
\text{Var}(R) - (\sigma_A^2 + \sigma_B^2) &= -2\rho\sigma_A\sigma_B
\end{align}

\textbf{Step 3: Condition for Reduction}
When $\rho > 0$:
\begin{align}
-2\rho\sigma_A\sigma_B < 0
\end{align}

Therefore:
\begin{align}
\text{Var}(R) < \sigma_A^2 + \sigma_B^2
\end{align}

\textbf{Step 4: Magnitude of Reduction}
The reduction is proportional to:
\begin{align}
\text{Reduction} &= 2\rho\sigma_A\sigma_B
\end{align}

\textbf{Conclusion:} Positive correlation reduces variance, with the reduction proportional to the correlation strength and the geometric mean of the standard deviations.

\subsection{Log-Transformation SNR Enhancement Proof}

\textbf{Theorem:} Under certain conditions, log-transformation can improve the signal-to-noise ratio for non-normal distributions.

\textbf{Proof:}

\textbf{Step 1: Log-Transformation Model}
For positive random variables $X_A, X_B$, define:
\begin{align}
Y_A &= \log(X_A) \\
Y_B &= \log(X_B)
\end{align}

\textbf{Step 2: Delta Method Approximation}
Using the delta method, for $Y = \log(X)$:
\begin{align}
E[Y] &\approx \log(E[X]) - \frac{\text{Var}(X)}{2E[X]^2} \\
\text{Var}(Y) &\approx \frac{\text{Var}(X)}{E[X]^2}
\end{align}

\textbf{Step 3: SNR Comparison}
Original SNR:
\begin{align}
\text{SNR}_{\text{original}} &= \frac{(\mu_A - \mu_B)^2}{\sigma_A^2}
\end{align}

Log-transformed SNR:
\begin{align}
\text{SNR}_{\text{log}} &\approx \frac{(\log(\mu_A) - \log(\mu_B))^2}{\sigma_A^2/\mu_A^2} \\
&= \frac{(\log(\mu_A/\mu_B))^2 \cdot \mu_A^2}{\sigma_A^2}
\end{align}

\textbf{Step 4: Enhancement Condition}
Log-transformation enhances SNR when:
\begin{align}
\frac{(\log(\mu_A/\mu_B))^2 \cdot \mu_A^2}{\sigma_A^2} > \frac{(\mu_A - \mu_B)^2}{\sigma_A^2}
\end{align}

Simplifying:
\begin{align}
(\log(\mu_A/\mu_B))^2 \cdot \mu_A^2 > (\mu_A - \mu_B)^2
\end{align}

\textbf{Conclusion:} Log-transformation enhances SNR when the relative difference in means is sufficiently large compared to the absolute difference, which is common for skewed distributions with high variance-to-mean ratios.

\subsection{Asymptote Analysis}

\textbf{Theorem:} The Signal Enhancement Factor (SEF) exhibits specific asymptotic behavior.

\textbf{Proof:}

\textbf{Case 1: $\rho \to 1$ (Perfect Positive Correlation)}
\begin{align}
\lim_{\rho \to 1} \text{SEF} &= \lim_{\rho \to 1} \frac{1 + \kappa}{1 + \kappa - 2\sqrt{\kappa}\rho} \\
&= \frac{1 + \kappa}{1 + \kappa - 2\sqrt{\kappa}} \\
&= \frac{1 + \kappa}{(\sqrt{\kappa} - 1)^2}
\end{align}

\textbf{Case 2: $\rho \to -1$ (Perfect Negative Correlation)}
\begin{align}
\lim_{\rho \to -1} \text{SEF} &= \lim_{\rho \to -1} \frac{1 + \kappa}{1 + \kappa - 2\sqrt{\kappa}\rho} \\
&= \frac{1 + \kappa}{1 + \kappa + 2\sqrt{\kappa}} \\
&= \frac{1 + \kappa}{(\sqrt{\kappa} + 1)^2} = 1
\end{align}

\textbf{Case 3: $\kappa \to 0$ (Team B Perfectly Consistent)}
\begin{align}
\lim_{\kappa \to 0} \text{SEF} &= \lim_{\kappa \to 0} \frac{1 + \kappa}{1 + \kappa - 2\sqrt{\kappa}\rho} \\
&= \frac{1}{1 - 0} = 1
\end{align}

\textbf{Case 4: $\kappa \to \infty$ (Team B Highly Variable)}
\begin{align}
\lim_{\kappa \to \infty} \text{SEF} &= \lim_{\kappa \to \infty} \frac{1 + \kappa}{1 + \kappa - 2\sqrt{\kappa}\rho} \\
&= \lim_{\kappa \to \infty} \frac{\kappa}{\kappa - 2\sqrt{\kappa}\rho} \\
&= \lim_{\kappa \to \infty} \frac{1}{1 - 2\rho/\sqrt{\kappa}} = 1
\end{align}

\textbf{Conclusion:} The asymptotic behavior confirms the theoretical bounds and provides insight into the framework's behavior under extreme conditions, following established enhancement factor analysis in signal processing literature.
